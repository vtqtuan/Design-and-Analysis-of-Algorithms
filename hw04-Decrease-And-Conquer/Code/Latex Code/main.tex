\documentclass[a4paper]{article} 
\input{head}
\usepackage{algorithm}
\usepackage{algorithmicx}
\usepackage{algpseudocode}
\usepackage{mathtools}
\usepackage{amsmath}
\DeclarePairedDelimiter\ceil{\lceil}{\rceil}
\DeclarePairedDelimiter\floor{\lfloor}{\rfloor}
\algdef{SE}[DOWHILE]{Do}{doWhile}{\algorithmicdo}[1]{\algorithmicwhile\ #1}%
\begin{document}

%-------------------------------
%	TITLE SECTION
%-------------------------------

\fancyhead[C]{}
\hrule \medskip % Upper rule
\begin{minipage}{0.295\textwidth} 
\raggedright
\footnotesize
VO TRAN QUANG TUAN\hfill\\   
18127248\hfill\\
18127248@student.hcmus.edu.vn
\end{minipage}
\begin{minipage}{0.4\textwidth} 
\centering 
\large 
Homework Assignment 4\\ 
\normalsize 
Introduction To Design And Analysis Of Algorithms, Fall 2020\\ 
\end{minipage}
\begin{minipage}{0.295\textwidth} 
\raggedleft
\today\hfill\\
\end{minipage}
\medskip\hrule 
\bigskip

%-------------------------------
%	CONTENTS
%-------------------------------
\tableofcontents
\newpage
%------------------------------------------------
\section{Problem 1}
By using divide-and-conquer approach, we can solve the change making problem with the time complexity is represented by the recurrence: 
\begin{gather*}
    T(n) = \sum_{i=1}^{\floor*{\frac{n}{2}}}[T(i) + T(n-i)] + \Theta(n), n >2 \\
    T(2)=1, T(1)=0
\end{gather*}
Replace $\Theta(n)$ by $\floor*{\frac{n}{2}}$. We have:
\begin{equation*}
    \begin{aligned}
        T(n) & = \sum_{i=1}^{\floor*{\frac{n}{2}}}[T(i) + T(n-i)] + \floor*{\frac{n}{2}} 
    \end{aligned}
\end{equation*}
\subsection{Consider the case $n$ is odd number}
With odd number, clearly: 
\begin{equation*}
    \begin{aligned}
        T(n) & = \sum_{i=1}^{\floor*{\frac{n}{2}}}[T(i) + T(n-i)] + \floor*{\frac{n}{2}} \\
        & = T(1)+T(2) + \ldots + T(\floor*{\frac{n}{2}}) + T(\floor*{\frac{n}{2}}+1) + \ldots + T(n-1) + \floor*{\frac{n}{2}}  \\ 
        & = \sum_{i=1}^{n-1}{T(i)} + \floor*{\frac{n}{2}} \\
        \implies 2 \cdot T(n) = 2 \cdot \sum_{i=1}^{n-1}{T(i)} + n 
    \end{aligned}
\end{equation*}
With $n$, from the equation above: 
\begin{equation}
    2 \cdot T(n) = 2 \cdot \sum_{i=1}^{n-1}{T(i)} + n 
\end{equation}
If replace $n$ by $n-1$, $n-1$ is even:
\begin{equation}
    2 \cdot T(n-1) = 2 \cdot \sum_{i=1}^{n-2}{T(i)} + n-1 + 2 \cdot T(\frac{n-1}{2})
\end{equation}
Subtract (2) from (1): 
\begin{equation*}
    \begin{aligned}
        2T(n) - 2T(n-1) = 2T(n-1) + 1 -  2 \cdot T(\frac{n-1}{2})
        & \Leftrightarrow T(n) = 2T(n-1) + 2^{-1} - T(\frac{n-1}{2})
    \end{aligned}
\end{equation*}
Comeback to the familiar form of recurrence equation, easily solve it by using backward substitution: 
\begin{equation*}
    \begin{aligned}
        T(n) & = 2T(n-1) + 2^{-1}  - T(\frac{n-1}{2}) \geq 3T(n-1)+2^{-1}\\
        & = 3 \cdot [3T(n-2) + 2^{-1}] + 2^{-1} = 3^2 \cdot T(n-2) + 3\cdot2^{-1} + 2^{-1} \\ 
        & = 3^2 \cdot[3T(n-3) + 2^{-1}] + 3\cdot2^{-1} + 2^{-1}  = 3^3 \cdot T(n-3) + 3^2\cdot2^{-1}+3\cdot2^{-1}+2^{-1} \\
        & = \ldots = 3^i \cdot T(n-i) + \frac{1}{2} \cdot \frac{3^i-1}{3-1} = 3^i \cdot T(n-i) + \frac{3^i-1}{4}
    \end{aligned}
\end{equation*}
Base on the initial condition $T(2)=1, T(1)=0$. \par 
If $n-i=1 \implies i=n-1$:
\begin{equation*}
    \begin{aligned}
        T(n) \geq 3^{n-1} \cdot 0 + \frac{3^{n-1}-1}{4}
        = \frac{3^{n-1}-1}{4} \in \Theta(3^n) \\
        \implies T(n) \in \Omega(2^n)
    \end{aligned}
\end{equation*}
If $n-i=2 \implies i=n-2$:
\begin{equation*}
    \begin{aligned}
        T(n) \geq 3^{n-2} \cdot 1 +  \frac{3^{n-2}-1}{4} \in \Theta(3^n) \\
        \implies T(n) \in \Omega(2^n)
    \end{aligned}
\end{equation*}
\subsection{Consider the case $n$ is even number}
With even number, clearly:
\begin{equation*}
    \begin{aligned}
        T(n) &= \sum_{i=1}^{\frac{n}{2}}[T(i)+T(n-i)] + \frac{n}{2} \\
        &= T(1)+T(2)+\ldots+T(\frac{n}{2})+T(\frac{n}{2})+\ldots+T(n-2)+T(n-1) + \frac{n}{2} \\ 
        &= \sum_{i=1}^{n-1}T(i) + T(\frac{n}{2}) + \frac{n}{2} 
    \end{aligned}
\end{equation*}
If replace $n$ by $n-1$, $n-1$ is odd:
\begin{equation*}
    2 \cdot T(n-1) = 2 \cdot \sum_{i=1}^{n-2}{T(i)} + n-1 
\end{equation*}
We can implies some equations like case 1:
\begin{equation*}
    \begin{aligned}
        2T(n) - 2T(n-1) = 2T(n-1) + 1 +  2 \cdot T(\frac{n}{2})
        & \Leftrightarrow T(n) = 2T(n-1) + 2^{-1} + T(\frac{n}{2})
    \end{aligned}
\end{equation*}
Comeback to the familiar form of recurrence equation, easily solve it by using backward substitution: 
\begin{equation*}
    \begin{aligned}
        T(n)& = 2T(n-1) + 2^{-1} + T(\frac{n}{2}) \geq 2T(n-1) + 2^{-1} \\
        & = 2 \cdot [2T(n-2) + 2^{-1}] + 2^{-1} = 2^2 \cdot T(n-2) + 2^0 + 2^{-1} \\ 
        & = 2^2 \cdot[2T(n-3) + 2^{-1}] + 2^0 + 2^{-1}  = 2^3 \cdot T(n-3) + 2^1+2^0+2^{-1} \\
        & = \ldots = 2^i \cdot T(n-i) + \frac{2^{i-1}-1}{2-1} + 2^{-1} = 2^i \cdot T(n-i) + 2^{i-1} - \frac{1}{2}
    \end{aligned}
\end{equation*}
Base on the initial condition $T(2)=1, T(1)=0$. \par 
If $n-i=1 \implies i=n-1$:
\begin{equation*}
    \begin{aligned}
        T(n) \geq 2^{n-1} \cdot 0 + 2^{n-2} - \frac{1}{2} 
        = 2^{n-2} - \frac{1}{2} \in \Theta(2^n) \\
        \implies T(n) \in \Omega(2^n)
    \end{aligned}
\end{equation*}
If $n-i=2 \implies i=n-2$:
\begin{equation*}
    \begin{aligned}
        T(n) \geq 2^{n-2} \cdot 1 + 2^{n-3} - \frac{1}{2}
         = 3 \cdot 2^{n-3} - \frac{1}{2} \in \Theta(2^n) \\
         \implies T(n) \in \Omega(2^n)
    \end{aligned}
\end{equation*}
\newpage
%------------------------------------------------
\section{Problem 2}
\newpage
%------------------------------------------------
\section{Problem 3}
Josephus problem
\subsection{Problem 3.a}
From the observation and analysis of Josephus problem, we have the recurrence: 
\begin{gather*}
    J(1)=1 \\ 
    n=2h, \;J(2h)=2J(h)-1 \\
    n=2h+1, \;J(2h+1)=2J(h)+1
\end{gather*}
A close-form solution to the two-case recurrence is as follows:
\begin{gather*}
    J(2^k+i)=2i+1, \; i \in [0, \ldots, 2^k-1]
\end{gather*}
With $k=0 \implies i=0$:
\begin{gather*}
    J(1) = 2 \cdot 0 + 1 = 1
\end{gather*}
With $2^k+i = 2m$, then:
\begin{equation*}
    \begin{aligned}
        J(2^k+i) &= 2J(2^{k-1}+\frac{i}{2})-1\\
        & = 2 \cdot [2 \cdot \frac{i}{2}+1]-1 = 2i+1
    \end{aligned}
\end{equation*}
With $2^k+i=2m+1$, then:
\begin{equation*}
    \begin{aligned}
        J(2^k+i) &= 2J(2^{k-1}+\frac{i-1}{2})+1 \\ 
        &= 2 \cdot [\frac{2 \cdot(i-1)}{2}+1]+1 \\
        &= 2i + 1
    \end{aligned}
\end{equation*}
The formula was proved.
\subsection{Problem 3.b}
We can represent $n = (b_mb_{m-1} \ldots b_1b_0)_2 = b_m \cdot 2^m + b_{m-1} \cdot2^{m-1} + \ldots + b_1 \cdot 2^1 + b_0 \cdot 2^0$. \par \bigskip
With $n=2^k+i \implies i = n-2^k = 0 \cdot 2^m + b_{m-1} \cdot2^{m-1} + \ldots + b_1 \cdot 2^1 + b_0 \cdot 2^0 = (0b_{m-1} \ldots b_1b_0)_2$.\par \bigskip
So, with $2i$, we have: $2i = (b_{m-1}b_{m-2} \ldots b_1b_00)_2$ \par \bigskip
From the problem 3.a, we proved: $J(2^k+1)=2i+1$:
\begin{equation*}
    \begin{aligned}
        \implies 2i+1& = (b_{m-1}b_{m-2} \ldots b_1b_01)_2 \\
        &= (b_{m-1}b_{m-2} \ldots b_1b_0b_m)_2
    \end{aligned}
\end{equation*}
We proved that: 
\begin{gather*}
    J[(b_mb_{m-1} \ldots b_1b_0)_2] = (b_{m-1}b_{m-2} \ldots b_1b_0b_m)_2
\end{gather*}
In conclusion, J(n) can be obtained by a 1-bit cyclic shift left of n itself.
\newpage
%------------------------------------------------
\section{Problem 4}
Bonus question: where does the following formula come from:
\begin{gather*}
    \sum_{i=1}^{n}{i \cdot 2^i} = (n-1) \cdot 2^{n+1} + 2
\end{gather*}
Clearly, expansion in series of the left side and replace $2$ by $x$: 
\begin{gather*}
     \sum_{i=1}^{n}{i \cdot x^i} = 1x^1 + 2x^2 + 3x^3 + \ldots nx^{n}
\end{gather*}
To begin with finding the general formula for that sum, we can see that we have $i \cdot x^i$. This observation makes it possible to connect in ideas is a familiar formula, let's call it $g(x)$:
\begin{gather*}
    g(x) = \frac{x^{n+1}-1}{x-1} = 1 + x^1+x^2+ \ldots + x^n
\end{gather*}
Take the derivative of $g(x)$: 
\begin{gather*}
    \dfrac{d}{dx}(\frac{x^{n+1}-1}{x-1})= 1 + 2x + 3x^2 + \ldots nx^{n-1}
\end{gather*}
Multiply $x$ by both sides of the equation: 
\begin{gather*}
    x \cdot \dfrac{d}{dx}(\frac{x^{n+1}-1}{x-1}) = 1x^1 + 2x^2 + 3x^3 + \ldots nx^{n} = \sum_{i=1}^{n}{i \cdot x^i}
\end{gather*}
Finding the general formula for this sum: 
\begin{equation*}
\begin{aligned}
    x \cdot \dfrac{d}{dx}(\frac{x^{n+1}-1}{x-1})& = x \cdot \frac{(n+1)x^n \cdot (x-1) - (x^{n+1}-1)}{(x-1)^2}  \\ 
    & = x \cdot \frac{nx^{n+1} - nx^n - x^n + 1}{(x-1)^2} \\
    & = \frac{x^{n+1} \cdot(nx -n-1) + x}{(x-1)^2}
\end{aligned}
\end{equation*}
Return $x=2$:
\begin{gather*}
    \frac{2^{n+1} \cdot(2n -n-1) + 2}{(2-1)^2} = (n-1) \cdot 2^{n+1} + 2
\end{gather*}
In conclusion: 
\begin{gather*}
     \sum_{i=1}^{n}{i \cdot 2^i} = (n-1) \cdot 2^{n+1} + 2
\end{gather*}
The formula in problem was proved.

\newpage
\end{document}
